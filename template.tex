% $Id: template.tex 11 2007-04-03 22:25:53Z jpeltier $

\documentclass{vgtc}                          % final (conference style)
%\documentclass[review]{vgtc}                 % review
%\documentclass[widereview]{vgtc}             % wide-spaced review
%\documentclass[preprint]{vgtc}               % preprint
%\documentclass[electronic]{vgtc}             % electronic version


%% Uncomment one of the lines above depending on where your paper is
%% in the conference process. ``review'' and ``widereview'' are for review
%% submission, ``preprint'' is for pre-publication, and the final version
%% doesn't use a specific qualifier. Further, ``electronic'' includes
%% hyperreferences for more convenient online viewing.

%% Please use one of the ``review'' options in combination with the
%% assigned online id (see below) ONLY if your paper uses a double blind
%% review process. Some conferences, like IEEE Vis and InfoVis, have NOT
%% in the past.

%% Figures should be in CMYK or Grey scale format, otherwise, colour 
%% shifting may occur during the printing process.

%% These few lines make a distinction between latex and pdflatex calls and they
%% bring in essential packages for graphics and font handling.
%% Note that due to the \DeclareGraphicsExtensions{} call it is no longer necessary
%% to provide the the path and extension of a graphics file:
%% \includegraphics{diamondrule} is completely sufficient.
%%
\ifpdf%                                % if we use pdflatex
  \pdfoutput=1\relax                   % create PDFs from pdfLaTeX
  \pdfcompresslevel=9                  % PDF Compression
  \pdfoptionpdfminorversion=7          % create PDF 1.7
  \ExecuteOptions{pdftex}
  \usepackage{graphicx}                % allow us to embed graphics files
  \DeclareGraphicsExtensions{.pdf,.png,.jpg,.jpeg} % for pdflatex we expect .pdf, .png, or .jpg files
\else%                                 % else we use pure latex
  \ExecuteOptions{dvips}
  \usepackage{graphicx}                % allow us to embed graphics files
  \DeclareGraphicsExtensions{.eps}     % for pure latex we expect eps files
\fi%

%% it is recomended to use ``\autoref{sec:bla}'' instead of ``Fig.~\ref{sec:bla}''
\graphicspath{{figures/}{pictures/}{images/}{./}} % where to search for the images

\usepackage{microtype}                 % use micro-typography (slightly more compact, better to read)
\PassOptionsToPackage{warn}{textcomp}  % to address font issues with \textrightarrow
\usepackage{textcomp}                  % use better special symbols
\usepackage{mathptmx}                  % use matching math font
\usepackage{times}                     % we use Times as the main font
\renewcommand*\ttdefault{txtt}         % a nicer typewriter font
\usepackage{cite}                      % needed to automatically sort the references
\usepackage{tabu}                      % only used for the table example
\usepackage{booktabs}                  % only used for the table example
%% We encourage the use of mathptmx for consistent usage of times font
%% throughout the proceedings. However, if you encounter conflicts
%% with other math-related packages, you may want to disable it.


%% If you are submitting a paper to a conference for review with a double
%% blind reviewing process, please replace the value ``0'' below with your
%% OnlineID. Otherwise, you may safely leave it at ``0''.
\onlineid{0}

%% declare the category of your paper, only shown in review mode
\vgtccategory{Research}

%% allow for this line if you want the electronic option to work properly
\vgtcinsertpkg

%% In preprint mode you may define your own headline.
%\preprinttext{To appear in an IEEE VGTC sponsored conference.}

%% Paper title.

\title{The accessibility of cities in metropolitan France}

%% This is how authors are specified in the conference style

%% Author and Affiliation (single author).
%%\author{Roy G. Biv\thanks{e-mail: roy.g.biv@aol.com}}
%%\affiliation{\scriptsize Allied Widgets Research}

%% Author and Affiliation (multiple authors with single affiliations).
%%\author{Roy G. Biv\thanks{e-mail: roy.g.biv@aol.com} %
%%\and Ed Grimley\thanks{e-mail:ed.grimley@aol.com} %
%%\and Martha Stewart\thanks{e-mail:martha.stewart@marthastewart.com}}
%%\affiliation{\scriptsize Martha Stewart Enterprises \\ Microsoft Research}

%% Author and Affiliation (multiple authors with multiple affiliations)
\author{Elia-Swarth \thanks{email:}\\ %
       %
\and Xian\thanks{e-mail: xian@aol.com}\\ %
      %
\and Aujogue Jean-baptiste\thanks{e-mail: jb.aujogue@gmail.com}} %
     
%% A teaser figure can be included as follows, but is not recommended since
%% the space is now taken up by a full width abstract.
%\teaser{
%  \includegraphics[width=1.5in]{sample.eps}
%  \caption{Lookit! Lookit!}
%}

%% Abstract section.
\abstract{The degree of accessibility of a city has an impact until its own identity, since it possesses deep effects on the demography, the activity and the culture of the city. In this work we propose a solution to visualize such an accessibility factor for a city, through the case of main cities of metropolitan France. After a brief presentation on existing studies of the socio-economic impact of city accessibility, we shall provide a detailed presentation of our visualization solution. We then conclude with a discussion on the possibilities to complete this visualization.
} % end of abstract

%% ACM Computing Classification System (CCS). 
%% See <http://www.acm.org/about/class> for details.
%% We recommend the 2012 system <http://www.acm.org/about/class/class/2012>
%% For the 2012 system use the ``\CCScatTwelve'' which command takes four arguments.
%% The 1998 system <http://www.acm.org/about/class/class/2012> is still possible
%% For the 1998 system use the ``\CCScat'' which command takes four arguments.
%% In both cases the last two arguments (1998) or last three (2012) can be empty.

\CCScatlist{
  \CCScatTwelve{Human-centered computing}{Visu\-al\-iza\-tion}{Visu\-al\-iza\-tion techniques}{Treemaps};
  \CCScatTwelve{Human-centered computing}{Visu\-al\-iza\-tion}{Visualization design and evaluation methods}{}
}

%\CCScatlist{
  %\CCScat{H.5.2}{User Interfaces}{User Interfaces}{Graphical user interfaces (GUI)}{};
  %\CCScat{H.5.m}{Information Interfaces and Presentation}{Miscellaneous}{}{}
%}

%% Copyright space is enabled by default as required by guidelines.
%% It is disabled by the 'review' option or via the following command:
% \nocopyrightspace

%%%%%%%%%%%%%%%%%%%%%%%%%%%%%%%%%%%%%%%%%%%%%%%%%%%%%%%%%%%%%%%%
%%%%%%%%%%%%%%%%%%%%%% START OF THE PAPER %%%%%%%%%%%%%%%%%%%%%%
%%%%%%%%%%%%%%%%%%%%%%%%%%%%%%%%%%%%%%%%%%%%%%%%%%%%%%%%%%%%%%%%%

\begin{document}

%% The ``\maketitle'' command must be the first command after the
%% ``\begin{document}'' command. It prepares and prints the title block.

%% the only exception to this rule is the \firstsection command
\firstsection{Introduction}

\maketitle
L'accessibilité d'une région au regard du monde extérieur a un impact énorme sur tous les aspects de la vie des habitants de cette région. Une facilité d'acces engrange un bassin d'attraction des cultures, des richesses et des savoir-faires. Ces phéniomènes faconnent naturellement l'identité de la région elle-meme, et de maniere générale tout le territoire.



Dans ce travail nous souhaitons étudier comment l'accessibilité se répartit sur le territoire de France métropolitaine. L'objectif est ici de présenter une synthese de cette accessibilité, ainsi que d'une présentation ville par ville. Une présentation sur carte sera l'objet central de cette etude, l'intéret principal étant qu'elle offre une répartition réaliste des villes sur le territoire ainsi qu'une lecture immédiate de l'information. 

Nous souhaitons également faire une étude comparative du mode de transport

enfin, il sera intérressant de présenter l'évolution temporelle de ces temps d'accès


Des représentations similaires pour le grand public sont accessibles \cite{LeMonde1}

\vspace{0.3cm}
\section{The story so far}

\vspace{0.3cm}

\subsection{Historique de l'étude de l'accessibilité au sein d'un territoire}

\vspace{0.3cm}
\subsection{Impact de l'accessibilité}

\vspace{0.3cm}



La mobilité au sein d'une population possède un impact sur de nombreux facteurs de cette population. Une facilité de déplacement admet généralement un impact positif sur les indicateurs économiques de cette population \cite{}. Ceci repose notamment sur la facilité de diffusion de connaissance, de culture et plus généralement de savoir-faire. Des effets bénéfiques notables sur la culture, la connaissance ainsi que sur le bien-etre ont également été observés. Une fluidité de déplacement possède enfin un impact sur l'homogénéisation de la population concernée.

augmentation de la productivité des entreprises (nouveaux liens entre acteurs, fluidification des échanges, réduction des couts, réduction de l'impact environnemental, créations d'emplois directs). Concentration de l'activité dans les centres névralgiques de réseaux. Une répartition uniforme des voies de transport au sein d'un territoire permet la formation de zones économiquement autonomes.

Une mauvaise accessibilité peut etre une raison suffisante pour un bachelier de ne pas effectuer ses études dans ladite région

L'accessibilité peut revetir une importance stratégique, qu'elle soit routiere, par chemin de fer ou par avion, pour attirer touristes, investisseurs et personnel compétent. 

L'étude des phénomenes de concentration de population (et donc de la puissance economique) en certains poles prend évidement en compte l'accessibilité de ces poles comme facteur d'accélération de cette concentration et l'excusivité de ces poles \cite{RePEc:mtp:titles:0262561476}.

\subsection{Travaux existants}

De nombreuses visualisation de l'état du traffic à grande échelle (et à petite échelle) sont disponibles, pour une discussion à ce propos on pourra consulter \cite{schoedon2016interactive}. 

La quantité d'articles traitant de l'impact des réseaux de transport sur la société est immense. Des revues entieres sont dédiées à ce sujet. Pour une lecture de certains aspects de l'impact de la géométrie du réseau de chemain de fer on pourra par exemple lire \cite{cao2017investigating} et les references mentionnées.


\section{Présentation de la visualisation}

\vspace{0.3cm}

\subsection{Acquisition de la donnée}

\vspace{0.3cm}

La donnée concernant les temps de trajet entre deux points du globe (d'une meme composante connexe) est extremmenent abondante, et fournie par un service de Google disponible sur l'API dédiée de Google \cite{APIGoogle}. Cette donnée s'obtient apres le lancement d'unre requete sous la forme d'une URL, dans laquelle est spécifiée le groupe de villes de départ, de villes d'arrivées, ainsi que le mode de transport (voiture/train) ainsi que la date à considérer. 

Chaque requete doit posséder un nombre séverement limité de villes de dépoart et d'arrivée (au plus dix). Face à cette contrainte, il a été nécessaire d'automatiser la procédure d'aquisition. Pour se faire on se base sur une liste des villes de france numérotées par population (disponible sur ), et on utilise un petit script python qui à un intervalle de numéros villes de départ, un intervalle de numéros de villes d'arrivées, un mode de transport et une date retourne l'URL qui fournit ces données. Chaque résultat de requete est alors stocké dans un fichier .json, et un autre script Python permet, basé sur l'ensemble de ces données aquises, d'associer à un numéro de ville de départ, de ville d'arrivée, de mode de transport et de date le temps de trajet souhaité. 

Les données de trajet explosent en le nombre de villes considérées, et ne sont en aucun cas stockable dans leur totalité. Cependant, une telle disponibilité de la donnée sur simple demande doit etre mise à profit. Une possibilité que nous souhaiterions aborder serait de mettre en place un champ dans la visualisation, qui permette a l'utilisateur de notre interface de visualiser l'accessibilité d'une ville supplémentaire de son choix.

\subsection{Structure de la visualisation}

\vspace{0.3cm}

La visualisation de l'accessibilité de chaque ville passe d'abord par une vue synthétique: Une seule carte du territoire, couverte par une surface ondulante, dont les pics correspondent aux zones les plus accessibles et les creux aux zones les moins accessibles. 

Le corps de la visualisation se présente également sous forme d'une carte du territoire, mais ou l'utilisateur est demandé de pointer une ville. Cette action déclenche une coloration de la carte, en nuances de gris ou par paliers de couleurs, qui permet de représenter le temps de trajet depuis n'importe quel point jusqu'à cette ville (ou inversement, depuis cette ville jusqu'à n'importe quel point de la carte). Les avantages d'une représentation par carte par rapport à d'autres solutions de visualisation sont multiples: On saisit en un instant la répartition des zones les plus accessibles depuis cette ville, et la comparaison avec la distance géographique réelle est immédiate. Dans cette visualisation deux options seront disponibles: Celle de choisir entre mode de transport (voiture et train), et celle de choisir la saison (été ou hiver). 

\begin{center}
INSERT VIZU HERE
\end{center}


Nous mettons également à disponibilité la possibilité de sélectionner les deux modes de transport: Dans ce cas, étant donnée une ville sélectionnée, la carte présente alors une partition bicolore du territoire, selon le mode de transport le plus avantageux pour se rendre en ce point. Là encore, une représentation en carte donne un comparatif immédiat du mode de transport à privilégier pour un déplacement.

\subsection{Implémentation}

\vspace{0.3cm}


La donnée disponible détermine le temps de trajet inter-ville parmis une liste établie de villes, et il est alors nécessaire d'étendre ceci afin de définir un temps de trajet entre deux points quelconques deu territoire. Une approximation de ceci peut etre obtenue en définissant un grille  Pour 
La connaissance essentielle dont nous avons besoin est le temps de trajet entre deux villes parmis une collection assez dense sur le territoire. De cette facon, nous pouvons 

\section{Présentations alternatives}



\section{Conclusion et perspectives}

\vspace{0.2cm}

Une étude précise de l'impact de l'accessibilité sur les indicateurs sur les indicateurs socio-économiques (tels que la concentration de population, le taux de diplomés, de chomage, la répartition de la population).
Un autre aspect d'intéret est de savoir qui voyage: la proportion de voyageurs selon des tranches d'age, de zone d'origine.
Enfin il serait intérressant de comparer l'impact socio-économique du degré de mobilité à l'échelle territoriala avec celle d'échelle péri-urbaine. 


%% if specified like this the section will be committed in review mode
%\acknowledgments{
%The authors wish to thank A, B, and C. This work was supported in part %by
%a grant from XYZ.}

%\bibliographystyle{abbrv}
\bibliographystyle{plain}
%\bibliographystyle{abbrv-doi-narrow}
%\bibliographystyle{abbrv-doi-hyperref}
%\bibliographystyle{abbrv-doi-hyperref-narrow}

\bibliography{biblio}
\end{document}
